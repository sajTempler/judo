
% Default to the notebook output style

    


% Inherit from the specified cell style.




    
\documentclass[11pt]{article}

    
    
    \usepackage[T1]{fontenc}
    % Nicer default font (+ math font) than Computer Modern for most use cases
    \usepackage{mathpazo}

    % Basic figure setup, for now with no caption control since it's done
    % automatically by Pandoc (which extracts ![](path) syntax from Markdown).
    \usepackage{graphicx}
    % We will generate all images so they have a width \maxwidth. This means
    % that they will get their normal width if they fit onto the page, but
    % are scaled down if they would overflow the margins.
    \makeatletter
    \def\maxwidth{\ifdim\Gin@nat@width>\linewidth\linewidth
    \else\Gin@nat@width\fi}
    \makeatother
    \let\Oldincludegraphics\includegraphics
    % Set max figure width to be 80% of text width, for now hardcoded.
    \renewcommand{\includegraphics}[1]{\Oldincludegraphics[width=.8\maxwidth]{#1}}
    % Ensure that by default, figures have no caption (until we provide a
    % proper Figure object with a Caption API and a way to capture that
    % in the conversion process - todo).
    \usepackage{caption}
    \DeclareCaptionLabelFormat{nolabel}{}
    \captionsetup{labelformat=nolabel}

    \usepackage{adjustbox} % Used to constrain images to a maximum size 
    \usepackage{xcolor} % Allow colors to be defined
    \usepackage{enumerate} % Needed for markdown enumerations to work
    \usepackage{geometry} % Used to adjust the document margins
    \usepackage{amsmath} % Equations
    \usepackage{amssymb} % Equations
    \usepackage{textcomp} % defines textquotesingle
    % Hack from http://tex.stackexchange.com/a/47451/13684:
    \AtBeginDocument{%
        \def\PYZsq{\textquotesingle}% Upright quotes in Pygmentized code
    }
    \usepackage{upquote} % Upright quotes for verbatim code
    \usepackage{eurosym} % defines \euro
    \usepackage[mathletters]{ucs} % Extended unicode (utf-8) support
    \usepackage[utf8x]{inputenc} % Allow utf-8 characters in the tex document
    \usepackage{fancyvrb} % verbatim replacement that allows latex
    \usepackage{grffile} % extends the file name processing of package graphics 
                         % to support a larger range 
    % The hyperref package gives us a pdf with properly built
    % internal navigation ('pdf bookmarks' for the table of contents,
    % internal cross-reference links, web links for URLs, etc.)
    \usepackage{hyperref}
    \usepackage{longtable} % longtable support required by pandoc >1.10
    \usepackage{booktabs}  % table support for pandoc > 1.12.2
    \usepackage[inline]{enumitem} % IRkernel/repr support (it uses the enumerate* environment)
    \usepackage[normalem]{ulem} % ulem is needed to support strikethroughs (\sout)
                                % normalem makes italics be italics, not underlines
    

    
    
    % Colors for the hyperref package
    \definecolor{urlcolor}{rgb}{0,.145,.698}
    \definecolor{linkcolor}{rgb}{.71,0.21,0.01}
    \definecolor{citecolor}{rgb}{.12,.54,.11}

    % ANSI colors
    \definecolor{ansi-black}{HTML}{3E424D}
    \definecolor{ansi-black-intense}{HTML}{282C36}
    \definecolor{ansi-red}{HTML}{E75C58}
    \definecolor{ansi-red-intense}{HTML}{B22B31}
    \definecolor{ansi-green}{HTML}{00A250}
    \definecolor{ansi-green-intense}{HTML}{007427}
    \definecolor{ansi-yellow}{HTML}{DDB62B}
    \definecolor{ansi-yellow-intense}{HTML}{B27D12}
    \definecolor{ansi-blue}{HTML}{208FFB}
    \definecolor{ansi-blue-intense}{HTML}{0065CA}
    \definecolor{ansi-magenta}{HTML}{D160C4}
    \definecolor{ansi-magenta-intense}{HTML}{A03196}
    \definecolor{ansi-cyan}{HTML}{60C6C8}
    \definecolor{ansi-cyan-intense}{HTML}{258F8F}
    \definecolor{ansi-white}{HTML}{C5C1B4}
    \definecolor{ansi-white-intense}{HTML}{A1A6B2}

    % commands and environments needed by pandoc snippets
    % extracted from the output of `pandoc -s`
    \providecommand{\tightlist}{%
      \setlength{\itemsep}{0pt}\setlength{\parskip}{0pt}}
    \DefineVerbatimEnvironment{Highlighting}{Verbatim}{commandchars=\\\{\}}
    % Add ',fontsize=\small' for more characters per line
    \newenvironment{Shaded}{}{}
    \newcommand{\KeywordTok}[1]{\textcolor[rgb]{0.00,0.44,0.13}{\textbf{{#1}}}}
    \newcommand{\DataTypeTok}[1]{\textcolor[rgb]{0.56,0.13,0.00}{{#1}}}
    \newcommand{\DecValTok}[1]{\textcolor[rgb]{0.25,0.63,0.44}{{#1}}}
    \newcommand{\BaseNTok}[1]{\textcolor[rgb]{0.25,0.63,0.44}{{#1}}}
    \newcommand{\FloatTok}[1]{\textcolor[rgb]{0.25,0.63,0.44}{{#1}}}
    \newcommand{\CharTok}[1]{\textcolor[rgb]{0.25,0.44,0.63}{{#1}}}
    \newcommand{\StringTok}[1]{\textcolor[rgb]{0.25,0.44,0.63}{{#1}}}
    \newcommand{\CommentTok}[1]{\textcolor[rgb]{0.38,0.63,0.69}{\textit{{#1}}}}
    \newcommand{\OtherTok}[1]{\textcolor[rgb]{0.00,0.44,0.13}{{#1}}}
    \newcommand{\AlertTok}[1]{\textcolor[rgb]{1.00,0.00,0.00}{\textbf{{#1}}}}
    \newcommand{\FunctionTok}[1]{\textcolor[rgb]{0.02,0.16,0.49}{{#1}}}
    \newcommand{\RegionMarkerTok}[1]{{#1}}
    \newcommand{\ErrorTok}[1]{\textcolor[rgb]{1.00,0.00,0.00}{\textbf{{#1}}}}
    \newcommand{\NormalTok}[1]{{#1}}
    
    % Additional commands for more recent versions of Pandoc
    \newcommand{\ConstantTok}[1]{\textcolor[rgb]{0.53,0.00,0.00}{{#1}}}
    \newcommand{\SpecialCharTok}[1]{\textcolor[rgb]{0.25,0.44,0.63}{{#1}}}
    \newcommand{\VerbatimStringTok}[1]{\textcolor[rgb]{0.25,0.44,0.63}{{#1}}}
    \newcommand{\SpecialStringTok}[1]{\textcolor[rgb]{0.73,0.40,0.53}{{#1}}}
    \newcommand{\ImportTok}[1]{{#1}}
    \newcommand{\DocumentationTok}[1]{\textcolor[rgb]{0.73,0.13,0.13}{\textit{{#1}}}}
    \newcommand{\AnnotationTok}[1]{\textcolor[rgb]{0.38,0.63,0.69}{\textbf{\textit{{#1}}}}}
    \newcommand{\CommentVarTok}[1]{\textcolor[rgb]{0.38,0.63,0.69}{\textbf{\textit{{#1}}}}}
    \newcommand{\VariableTok}[1]{\textcolor[rgb]{0.10,0.09,0.49}{{#1}}}
    \newcommand{\ControlFlowTok}[1]{\textcolor[rgb]{0.00,0.44,0.13}{\textbf{{#1}}}}
    \newcommand{\OperatorTok}[1]{\textcolor[rgb]{0.40,0.40,0.40}{{#1}}}
    \newcommand{\BuiltInTok}[1]{{#1}}
    \newcommand{\ExtensionTok}[1]{{#1}}
    \newcommand{\PreprocessorTok}[1]{\textcolor[rgb]{0.74,0.48,0.00}{{#1}}}
    \newcommand{\AttributeTok}[1]{\textcolor[rgb]{0.49,0.56,0.16}{{#1}}}
    \newcommand{\InformationTok}[1]{\textcolor[rgb]{0.38,0.63,0.69}{\textbf{\textit{{#1}}}}}
    \newcommand{\WarningTok}[1]{\textcolor[rgb]{0.38,0.63,0.69}{\textbf{\textit{{#1}}}}}
    
    
    % Define a nice break command that doesn't care if a line doesn't already
    % exist.
    \def\br{\hspace*{\fill} \\* }
    % Math Jax compatability definitions
    \def\gt{>}
    \def\lt{<}
    % Document parameters
    \title{first\_info}
    
    
    

    % Pygments definitions
    
\makeatletter
\def\PY@reset{\let\PY@it=\relax \let\PY@bf=\relax%
    \let\PY@ul=\relax \let\PY@tc=\relax%
    \let\PY@bc=\relax \let\PY@ff=\relax}
\def\PY@tok#1{\csname PY@tok@#1\endcsname}
\def\PY@toks#1+{\ifx\relax#1\empty\else%
    \PY@tok{#1}\expandafter\PY@toks\fi}
\def\PY@do#1{\PY@bc{\PY@tc{\PY@ul{%
    \PY@it{\PY@bf{\PY@ff{#1}}}}}}}
\def\PY#1#2{\PY@reset\PY@toks#1+\relax+\PY@do{#2}}

\expandafter\def\csname PY@tok@w\endcsname{\def\PY@tc##1{\textcolor[rgb]{0.73,0.73,0.73}{##1}}}
\expandafter\def\csname PY@tok@c\endcsname{\let\PY@it=\textit\def\PY@tc##1{\textcolor[rgb]{0.25,0.50,0.50}{##1}}}
\expandafter\def\csname PY@tok@cp\endcsname{\def\PY@tc##1{\textcolor[rgb]{0.74,0.48,0.00}{##1}}}
\expandafter\def\csname PY@tok@k\endcsname{\let\PY@bf=\textbf\def\PY@tc##1{\textcolor[rgb]{0.00,0.50,0.00}{##1}}}
\expandafter\def\csname PY@tok@kp\endcsname{\def\PY@tc##1{\textcolor[rgb]{0.00,0.50,0.00}{##1}}}
\expandafter\def\csname PY@tok@kt\endcsname{\def\PY@tc##1{\textcolor[rgb]{0.69,0.00,0.25}{##1}}}
\expandafter\def\csname PY@tok@o\endcsname{\def\PY@tc##1{\textcolor[rgb]{0.40,0.40,0.40}{##1}}}
\expandafter\def\csname PY@tok@ow\endcsname{\let\PY@bf=\textbf\def\PY@tc##1{\textcolor[rgb]{0.67,0.13,1.00}{##1}}}
\expandafter\def\csname PY@tok@nb\endcsname{\def\PY@tc##1{\textcolor[rgb]{0.00,0.50,0.00}{##1}}}
\expandafter\def\csname PY@tok@nf\endcsname{\def\PY@tc##1{\textcolor[rgb]{0.00,0.00,1.00}{##1}}}
\expandafter\def\csname PY@tok@nc\endcsname{\let\PY@bf=\textbf\def\PY@tc##1{\textcolor[rgb]{0.00,0.00,1.00}{##1}}}
\expandafter\def\csname PY@tok@nn\endcsname{\let\PY@bf=\textbf\def\PY@tc##1{\textcolor[rgb]{0.00,0.00,1.00}{##1}}}
\expandafter\def\csname PY@tok@ne\endcsname{\let\PY@bf=\textbf\def\PY@tc##1{\textcolor[rgb]{0.82,0.25,0.23}{##1}}}
\expandafter\def\csname PY@tok@nv\endcsname{\def\PY@tc##1{\textcolor[rgb]{0.10,0.09,0.49}{##1}}}
\expandafter\def\csname PY@tok@no\endcsname{\def\PY@tc##1{\textcolor[rgb]{0.53,0.00,0.00}{##1}}}
\expandafter\def\csname PY@tok@nl\endcsname{\def\PY@tc##1{\textcolor[rgb]{0.63,0.63,0.00}{##1}}}
\expandafter\def\csname PY@tok@ni\endcsname{\let\PY@bf=\textbf\def\PY@tc##1{\textcolor[rgb]{0.60,0.60,0.60}{##1}}}
\expandafter\def\csname PY@tok@na\endcsname{\def\PY@tc##1{\textcolor[rgb]{0.49,0.56,0.16}{##1}}}
\expandafter\def\csname PY@tok@nt\endcsname{\let\PY@bf=\textbf\def\PY@tc##1{\textcolor[rgb]{0.00,0.50,0.00}{##1}}}
\expandafter\def\csname PY@tok@nd\endcsname{\def\PY@tc##1{\textcolor[rgb]{0.67,0.13,1.00}{##1}}}
\expandafter\def\csname PY@tok@s\endcsname{\def\PY@tc##1{\textcolor[rgb]{0.73,0.13,0.13}{##1}}}
\expandafter\def\csname PY@tok@sd\endcsname{\let\PY@it=\textit\def\PY@tc##1{\textcolor[rgb]{0.73,0.13,0.13}{##1}}}
\expandafter\def\csname PY@tok@si\endcsname{\let\PY@bf=\textbf\def\PY@tc##1{\textcolor[rgb]{0.73,0.40,0.53}{##1}}}
\expandafter\def\csname PY@tok@se\endcsname{\let\PY@bf=\textbf\def\PY@tc##1{\textcolor[rgb]{0.73,0.40,0.13}{##1}}}
\expandafter\def\csname PY@tok@sr\endcsname{\def\PY@tc##1{\textcolor[rgb]{0.73,0.40,0.53}{##1}}}
\expandafter\def\csname PY@tok@ss\endcsname{\def\PY@tc##1{\textcolor[rgb]{0.10,0.09,0.49}{##1}}}
\expandafter\def\csname PY@tok@sx\endcsname{\def\PY@tc##1{\textcolor[rgb]{0.00,0.50,0.00}{##1}}}
\expandafter\def\csname PY@tok@m\endcsname{\def\PY@tc##1{\textcolor[rgb]{0.40,0.40,0.40}{##1}}}
\expandafter\def\csname PY@tok@gh\endcsname{\let\PY@bf=\textbf\def\PY@tc##1{\textcolor[rgb]{0.00,0.00,0.50}{##1}}}
\expandafter\def\csname PY@tok@gu\endcsname{\let\PY@bf=\textbf\def\PY@tc##1{\textcolor[rgb]{0.50,0.00,0.50}{##1}}}
\expandafter\def\csname PY@tok@gd\endcsname{\def\PY@tc##1{\textcolor[rgb]{0.63,0.00,0.00}{##1}}}
\expandafter\def\csname PY@tok@gi\endcsname{\def\PY@tc##1{\textcolor[rgb]{0.00,0.63,0.00}{##1}}}
\expandafter\def\csname PY@tok@gr\endcsname{\def\PY@tc##1{\textcolor[rgb]{1.00,0.00,0.00}{##1}}}
\expandafter\def\csname PY@tok@ge\endcsname{\let\PY@it=\textit}
\expandafter\def\csname PY@tok@gs\endcsname{\let\PY@bf=\textbf}
\expandafter\def\csname PY@tok@gp\endcsname{\let\PY@bf=\textbf\def\PY@tc##1{\textcolor[rgb]{0.00,0.00,0.50}{##1}}}
\expandafter\def\csname PY@tok@go\endcsname{\def\PY@tc##1{\textcolor[rgb]{0.53,0.53,0.53}{##1}}}
\expandafter\def\csname PY@tok@gt\endcsname{\def\PY@tc##1{\textcolor[rgb]{0.00,0.27,0.87}{##1}}}
\expandafter\def\csname PY@tok@err\endcsname{\def\PY@bc##1{\setlength{\fboxsep}{0pt}\fcolorbox[rgb]{1.00,0.00,0.00}{1,1,1}{\strut ##1}}}
\expandafter\def\csname PY@tok@kc\endcsname{\let\PY@bf=\textbf\def\PY@tc##1{\textcolor[rgb]{0.00,0.50,0.00}{##1}}}
\expandafter\def\csname PY@tok@kd\endcsname{\let\PY@bf=\textbf\def\PY@tc##1{\textcolor[rgb]{0.00,0.50,0.00}{##1}}}
\expandafter\def\csname PY@tok@kn\endcsname{\let\PY@bf=\textbf\def\PY@tc##1{\textcolor[rgb]{0.00,0.50,0.00}{##1}}}
\expandafter\def\csname PY@tok@kr\endcsname{\let\PY@bf=\textbf\def\PY@tc##1{\textcolor[rgb]{0.00,0.50,0.00}{##1}}}
\expandafter\def\csname PY@tok@bp\endcsname{\def\PY@tc##1{\textcolor[rgb]{0.00,0.50,0.00}{##1}}}
\expandafter\def\csname PY@tok@fm\endcsname{\def\PY@tc##1{\textcolor[rgb]{0.00,0.00,1.00}{##1}}}
\expandafter\def\csname PY@tok@vc\endcsname{\def\PY@tc##1{\textcolor[rgb]{0.10,0.09,0.49}{##1}}}
\expandafter\def\csname PY@tok@vg\endcsname{\def\PY@tc##1{\textcolor[rgb]{0.10,0.09,0.49}{##1}}}
\expandafter\def\csname PY@tok@vi\endcsname{\def\PY@tc##1{\textcolor[rgb]{0.10,0.09,0.49}{##1}}}
\expandafter\def\csname PY@tok@vm\endcsname{\def\PY@tc##1{\textcolor[rgb]{0.10,0.09,0.49}{##1}}}
\expandafter\def\csname PY@tok@sa\endcsname{\def\PY@tc##1{\textcolor[rgb]{0.73,0.13,0.13}{##1}}}
\expandafter\def\csname PY@tok@sb\endcsname{\def\PY@tc##1{\textcolor[rgb]{0.73,0.13,0.13}{##1}}}
\expandafter\def\csname PY@tok@sc\endcsname{\def\PY@tc##1{\textcolor[rgb]{0.73,0.13,0.13}{##1}}}
\expandafter\def\csname PY@tok@dl\endcsname{\def\PY@tc##1{\textcolor[rgb]{0.73,0.13,0.13}{##1}}}
\expandafter\def\csname PY@tok@s2\endcsname{\def\PY@tc##1{\textcolor[rgb]{0.73,0.13,0.13}{##1}}}
\expandafter\def\csname PY@tok@sh\endcsname{\def\PY@tc##1{\textcolor[rgb]{0.73,0.13,0.13}{##1}}}
\expandafter\def\csname PY@tok@s1\endcsname{\def\PY@tc##1{\textcolor[rgb]{0.73,0.13,0.13}{##1}}}
\expandafter\def\csname PY@tok@mb\endcsname{\def\PY@tc##1{\textcolor[rgb]{0.40,0.40,0.40}{##1}}}
\expandafter\def\csname PY@tok@mf\endcsname{\def\PY@tc##1{\textcolor[rgb]{0.40,0.40,0.40}{##1}}}
\expandafter\def\csname PY@tok@mh\endcsname{\def\PY@tc##1{\textcolor[rgb]{0.40,0.40,0.40}{##1}}}
\expandafter\def\csname PY@tok@mi\endcsname{\def\PY@tc##1{\textcolor[rgb]{0.40,0.40,0.40}{##1}}}
\expandafter\def\csname PY@tok@il\endcsname{\def\PY@tc##1{\textcolor[rgb]{0.40,0.40,0.40}{##1}}}
\expandafter\def\csname PY@tok@mo\endcsname{\def\PY@tc##1{\textcolor[rgb]{0.40,0.40,0.40}{##1}}}
\expandafter\def\csname PY@tok@ch\endcsname{\let\PY@it=\textit\def\PY@tc##1{\textcolor[rgb]{0.25,0.50,0.50}{##1}}}
\expandafter\def\csname PY@tok@cm\endcsname{\let\PY@it=\textit\def\PY@tc##1{\textcolor[rgb]{0.25,0.50,0.50}{##1}}}
\expandafter\def\csname PY@tok@cpf\endcsname{\let\PY@it=\textit\def\PY@tc##1{\textcolor[rgb]{0.25,0.50,0.50}{##1}}}
\expandafter\def\csname PY@tok@c1\endcsname{\let\PY@it=\textit\def\PY@tc##1{\textcolor[rgb]{0.25,0.50,0.50}{##1}}}
\expandafter\def\csname PY@tok@cs\endcsname{\let\PY@it=\textit\def\PY@tc##1{\textcolor[rgb]{0.25,0.50,0.50}{##1}}}

\def\PYZbs{\char`\\}
\def\PYZus{\char`\_}
\def\PYZob{\char`\{}
\def\PYZcb{\char`\}}
\def\PYZca{\char`\^}
\def\PYZam{\char`\&}
\def\PYZlt{\char`\<}
\def\PYZgt{\char`\>}
\def\PYZsh{\char`\#}
\def\PYZpc{\char`\%}
\def\PYZdl{\char`\$}
\def\PYZhy{\char`\-}
\def\PYZsq{\char`\'}
\def\PYZdq{\char`\"}
\def\PYZti{\char`\~}
% for compatibility with earlier versions
\def\PYZat{@}
\def\PYZlb{[}
\def\PYZrb{]}
\makeatother


    % Exact colors from NB
    \definecolor{incolor}{rgb}{0.0, 0.0, 0.5}
    \definecolor{outcolor}{rgb}{0.545, 0.0, 0.0}



    
    % Prevent overflowing lines due to hard-to-break entities
    \sloppy 
    % Setup hyperref package
    \hypersetup{
      breaklinks=true,  % so long urls are correctly broken across lines
      colorlinks=true,
      urlcolor=urlcolor,
      linkcolor=linkcolor,
      citecolor=citecolor,
      }
    % Slightly bigger margins than the latex defaults
    
    \geometry{verbose,tmargin=1in,bmargin=1in,lmargin=1in,rmargin=1in}
    
    

    \begin{document}
    
    
    \maketitle
    
    

    
    \subsection{Un primer boceto de la info que hay en
judobase}\label{un-primer-boceto-de-la-info-que-hay-en-judobase}

    \subsubsection{Importar funciones}\label{importar-funciones}

    Todo ha sido descargado previamente y almacenado en una base de datos:

    \begin{Verbatim}[commandchars=\\\{\}]
{\color{incolor}In [{\color{incolor}1}]:} \PY{k+kn}{from} \PY{n+nn}{lib} \PY{k}{import} \PY{n}{SQLiteConnection}
        
        \PY{k+kn}{import} \PY{n+nn}{pandas} \PY{k}{as} \PY{n+nn}{pd}
        \PY{n}{pd}\PY{o}{.}\PY{n}{options}\PY{o}{.}\PY{n}{display}\PY{o}{.}\PY{n}{max\PYZus{}colwidth} \PY{o}{=} \PY{l+m+mi}{100}
\end{Verbatim}


    \begin{Verbatim}[commandchars=\\\{\}]
{\color{incolor}In [{\color{incolor}2}]:} \PY{n}{conn} \PY{o}{=} \PY{n}{SQLiteConnection}\PY{p}{(}\PY{l+s+s1}{\PYZsq{}}\PY{l+s+s1}{../judo.sql}\PY{l+s+s1}{\PYZsq{}}\PY{p}{)}
\end{Verbatim}


    \subsubsection{Battles Estrella}\label{battles-estrella}

    Dataframe que contiene la info de batallas

    \begin{Verbatim}[commandchars=\\\{\}]
{\color{incolor}In [{\color{incolor}3}]:} \PY{n}{df} \PY{o}{=} \PY{n}{conn}\PY{o}{.}\PY{n}{as\PYZus{}pandas}\PY{p}{(}\PY{l+s+s1}{\PYZsq{}}\PY{l+s+s1}{select * from estre}\PY{l+s+s1}{\PYZsq{}}\PY{p}{)}
\end{Verbatim}


    Cada fila es una acción en un combate, donde:

\textbf{you} = 1 significa que es tuyo (el shido, wazari...)\\
\textbf{you} = 0 significa que es del oponente

    \begin{Verbatim}[commandchars=\\\{\}]
{\color{incolor}In [{\color{incolor}4}]:} \PY{n}{df}\PY{o}{.}\PY{n}{head}\PY{p}{(}\PY{l+m+mi}{10}\PY{p}{)}
\end{Verbatim}


\begin{Verbatim}[commandchars=\\\{\}]
{\color{outcolor}Out[{\color{outcolor}4}]:}            opponent  you    action            action\_detail   time  \textbackslash{}
        0       PARK Da Sol    1     Shido     Shido / False-Attack  00:15   
        1       PARK Da Sol    0     Shido     Shido / False-Attack  01:29   
        2       PARK Da Sol    1     Shido  Shido / Non-Combativity  03:38   
        3       PARK Da Sol    0     Ippon                           03:46   
        4         PUPP Reka    1     Shido  Shido / Non-Combativity  01:22   
        5         PUPP Reka    1     Shido  Shido / Non-Combativity  03:05   
        6         PUPP Reka    0  Waza-ari                           04:00   
        7  JIMENEZ Kristine    1  Waza-ari                           00:09   
        8  JIMENEZ Kristine    0     Shido  Shido / Non-Combativity  02:36   
        9     GNETO Astride    1     Shido     Shido / False-Attack  00:31   
        
                                                                     url\_youtube  \textbackslash{}
        0  https://judobase.ijf.org/\#/competition/contest/gs\_jpn2018\_w\_0052\_0012   
        1  https://judobase.ijf.org/\#/competition/contest/gs\_jpn2018\_w\_0052\_0012   
        2  https://judobase.ijf.org/\#/competition/contest/gs\_jpn2018\_w\_0052\_0012   
        3  https://judobase.ijf.org/\#/competition/contest/gs\_jpn2018\_w\_0052\_0012   
        4  https://judobase.ijf.org/\#/competition/contest/gp\_ned2018\_w\_0052\_0023   
        5  https://judobase.ijf.org/\#/competition/contest/gp\_ned2018\_w\_0052\_0023   
        6  https://judobase.ijf.org/\#/competition/contest/gp\_ned2018\_w\_0052\_0023   
        7  https://judobase.ijf.org/\#/competition/contest/gp\_ned2018\_w\_0052\_0014   
        8  https://judobase.ijf.org/\#/competition/contest/gp\_ned2018\_w\_0052\_0014   
        9  https://judobase.ijf.org/\#/competition/contest/gs\_ger2018\_w\_0052\_0033   
        
                                event  
        0       Grand Slam Osaka 2018  
        1       Grand Slam Osaka 2018  
        2       Grand Slam Osaka 2018  
        3       Grand Slam Osaka 2018  
        4   Grand Prix The Hague 2018  
        5   Grand Prix The Hague 2018  
        6   Grand Prix The Hague 2018  
        7   Grand Prix The Hague 2018  
        8   Grand Prix The Hague 2018  
        9  Grand Slam Dusseldorf 2018  
\end{Verbatim}
            
    \begin{Verbatim}[commandchars=\\\{\}]
{\color{incolor}In [{\color{incolor}5}]:} \PY{n+nb}{print}\PY{p}{(}\PY{n}{f}\PY{l+s+s1}{\PYZsq{}}\PY{l+s+s1}{Es decir, un total de }\PY{l+s+si}{\PYZob{}df.shape[0]\PYZcb{}}\PY{l+s+s1}{ acciones en }\PY{l+s+s1}{\PYZob{}}\PY{l+s+s1}{df[[}\PY{l+s+s1}{\PYZdq{}}\PY{l+s+s1}{opponent}\PY{l+s+s1}{\PYZdq{}}\PY{l+s+s1}{, }\PY{l+s+s1}{\PYZdq{}}\PY{l+s+s1}{event}\PY{l+s+s1}{\PYZdq{}}\PY{l+s+s1}{]].drop\PYZus{}duplicates().shape[0]\PYZcb{} batallas}\PY{l+s+s1}{\PYZsq{}}\PY{p}{)}
\end{Verbatim}


    \begin{Verbatim}[commandchars=\\\{\}]
Es decir, un total de 82 acciones en 23 batallas

    \end{Verbatim}

    Agrupamos por sujeto y acción

    \begin{Verbatim}[commandchars=\\\{\}]
{\color{incolor}In [{\color{incolor}6}]:} \PY{n}{grouped\PYZus{}actions} \PY{o}{=} \PY{n}{df}\PY{o}{.}\PY{n}{groupby}\PY{p}{(}\PY{p}{[}\PY{l+s+s1}{\PYZsq{}}\PY{l+s+s1}{you}\PY{l+s+s1}{\PYZsq{}}\PY{p}{,} \PY{l+s+s1}{\PYZsq{}}\PY{l+s+s1}{action}\PY{l+s+s1}{\PYZsq{}}\PY{p}{]}\PY{p}{)}\PY{o}{.}\PY{n}{size}\PY{p}{(}\PY{p}{)}
        \PY{n}{grouped\PYZus{}actions}
\end{Verbatim}


\begin{Verbatim}[commandchars=\\\{\}]
{\color{outcolor}Out[{\color{outcolor}6}]:} you  action                
        0    Cancel Waza-ari            2
             HSK (3rd shido)            2
             Ippon                      4
             Shido                     22
             Waza-Ari-awasete-ippon     2
             Waza-ari                  16
        1    Ippon                      3
             Shido                     21
             Waza-Ari-awasete-ippon     1
             Waza-ari                   9
        dtype: int64
\end{Verbatim}
            
    \begin{Verbatim}[commandchars=\\\{\}]
{\color{incolor}In [{\color{incolor}7}]:} \PY{n+nb}{print}\PY{p}{(}\PY{n}{f}\PY{l+s+s1}{\PYZsq{}}\PY{l+s+s1}{Por ejemplo, hiciste }\PY{l+s+si}{\PYZob{}grouped\PYZus{}actions[1, \PYZdq{}Ippon\PYZdq{}]\PYZcb{}}\PY{l+s+s1}{ Ippones y recibiste }\PY{l+s+si}{\PYZob{}grouped\PYZus{}actions[1, \PYZdq{}Shido\PYZdq{}]\PYZcb{}}\PY{l+s+s1}{ Shidos}\PY{l+s+s1}{\PYZsq{}}\PY{p}{)}
\end{Verbatim}


    \begin{Verbatim}[commandchars=\\\{\}]
Por ejemplo, hiciste 3 Ippones y recibiste 21 Shidos

    \end{Verbatim}

    \subsubsection{Detail Estrella}\label{detail-estrella}

    \begin{Verbatim}[commandchars=\\\{\}]
{\color{incolor}In [{\color{incolor}8}]:} \PY{n}{df}\PY{o}{.}\PY{n}{groupby}\PY{p}{(}\PY{p}{[}\PY{l+s+s1}{\PYZsq{}}\PY{l+s+s1}{you}\PY{l+s+s1}{\PYZsq{}}\PY{p}{,} \PY{l+s+s1}{\PYZsq{}}\PY{l+s+s1}{action\PYZus{}detail}\PY{l+s+s1}{\PYZsq{}}\PY{p}{]}\PY{p}{)}\PY{o}{.}\PY{n}{size}\PY{p}{(}\PY{p}{)}
\end{Verbatim}


\begin{Verbatim}[commandchars=\\\{\}]
{\color{outcolor}Out[{\color{outcolor}8}]:} you  action\_detail                    
        0                                          4
             Ashi-waza / Hiza-guruma               1
             Ashi-waza / O-soto-gaeshi             1
             Ashi-waza / O-soto-gari               2
             Ashi-waza / O-uchi-gaeshi             1
             Ashi-waza / O-uchi-gari               3
             Ashi-waza / Okuri-ashi-harai          1
             Ashi-waza / Uchi-mata                 1
             Cancel Yuko                           1
             Ma-sutemi-waza / Sumi-gaeshi          1
             Ma-sutemi-waza / Tomoe-nage           1
             Osaekomi-waza / Kesa-gatame           1
             Shido / Avoid-Grip                    4
             Shido / Defensive-Posture             4
             Shido / False-Attack                  4
             Shido / Hold-Sleeve-Ends              2
             Shido / Hold-Trouser-Leg              1
             Shido / Non-Combativity               7
             Shido / Outside-Contest-Area          2
             Te-waza / Ippon-seoi-nage             1
             Te-waza / Seoi-nage                   2
             Te-waza / Sumi-otoshi                 1
             Yoko-sutemi-waza / Harai-makikomi     1
             Yoko-sutemi-waza / Tani-otoshi        1
        1                                          1
             Ashi-waza / O-soto-gaeshi             1
             Ashi-waza / Sasae-tsurikomi-ashi      1
             Ma-sutemi-waza / Sumi-gaeshi          2
             Osaekomi-waza / Yoko-shiho-gatame     2
             Shido / Avoid-Grip                    1
             Shido / Bend-Opps-Fingers             1
             Shido / Escape-With-Head              1
             Shido / False-Attack                  7
             Shido / Hold-Same-Side                1
             Shido / Non-Combativity              10
             Te-waza / Ippon-seoi-nage             2
             Te-waza / Ko-uchi-gaeshi              1
             Te-waza / Sumi-otoshi                 1
             Te-waza / Uchi-mata-sukashi           1
             Yoko-sutemi-waza / Harai-makikomi     1
        dtype: int64
\end{Verbatim}
            
    \subsubsection{Battles Gaitero}\label{battles-gaitero}

    \begin{Verbatim}[commandchars=\\\{\}]
{\color{incolor}In [{\color{incolor}9}]:} \PY{c+c1}{\PYZsh{} dataframe containing battles info}
        \PY{n}{df} \PY{o}{=} \PY{n}{conn}\PY{o}{.}\PY{n}{as\PYZus{}pandas}\PY{p}{(}\PY{l+s+s1}{\PYZsq{}}\PY{l+s+s1}{select * from gaite}\PY{l+s+s1}{\PYZsq{}}\PY{p}{)}
\end{Verbatim}


    Cada fila es una acción en un combate, donde:

\textbf{you} = 1 significa que es tuyo (el shido, wazari...)\\
\textbf{you} = 0 significa que es del oponente

    \begin{Verbatim}[commandchars=\\\{\}]
{\color{incolor}In [{\color{incolor}10}]:} \PY{n}{df}\PY{o}{.}\PY{n}{head}\PY{p}{(}\PY{l+m+mi}{10}\PY{p}{)}
\end{Verbatim}


\begin{Verbatim}[commandchars=\\\{\}]
{\color{outcolor}Out[{\color{outcolor}10}]:}            opponent  you        action                       action\_detail  \textbackslash{}
         0   SHMAILOV Baruch    1         Shido                Shido / False-Attack   
         1   SHMAILOV Baruch    0         Shido                Shido / False-Attack   
         2   SHMAILOV Baruch    1         Shido            Shido / Hold-Trouser-Leg   
         3   SHMAILOV Baruch    0         Ippon               Ashi-waza / Uchi-mata   
         4        BAI Zhijie    1         Shido                  Shido / Avoid-Grip   
         5        BAI Zhijie    1         Shido         Shido / Fingers-interlocked   
         6        BAI Zhijie    1      Waza-ari   Koshi-waza / Sode-tsurikomi-goshi   
         7  SHERSHAN Dzmitry    1         Shido             Shido / Non-Combativity   
         8  SHERSHAN Dzmitry    1  Cancel Ippon                                       
         9  SHERSHAN Dzmitry    0         Ippon  Yoko-sutemi-waza / O-soto-makikomi   
         
             time  \textbackslash{}
         0  00:22   
         1  01:28   
         2  03:55   
         3  04:23   
         4  01:34   
         5  02:10   
         6  02:42   
         7  04:50   
         8  05:31   
         9  08:43   
         
                                                                      url\_youtube  \textbackslash{}
         0  https://judobase.ijf.org/\#/competition/contest/gp\_ned2018\_m\_0066\_0055   
         1  https://judobase.ijf.org/\#/competition/contest/gp\_ned2018\_m\_0066\_0055   
         2  https://judobase.ijf.org/\#/competition/contest/gp\_ned2018\_m\_0066\_0055   
         3  https://judobase.ijf.org/\#/competition/contest/gp\_ned2018\_m\_0066\_0055   
         4  https://judobase.ijf.org/\#/competition/contest/gp\_ned2018\_m\_0066\_0046   
         5  https://judobase.ijf.org/\#/competition/contest/gp\_ned2018\_m\_0066\_0046   
         6  https://judobase.ijf.org/\#/competition/contest/gp\_ned2018\_m\_0066\_0046   
         7  https://judobase.ijf.org/\#/competition/contest/gs\_uae2018\_m\_0066\_0033   
         8  https://judobase.ijf.org/\#/competition/contest/gs\_uae2018\_m\_0066\_0033   
         9  https://judobase.ijf.org/\#/competition/contest/gs\_uae2018\_m\_0066\_0033   
         
                                event  
         0  Grand Prix The Hague 2018  
         1  Grand Prix The Hague 2018  
         2  Grand Prix The Hague 2018  
         3  Grand Prix The Hague 2018  
         4  Grand Prix The Hague 2018  
         5  Grand Prix The Hague 2018  
         6  Grand Prix The Hague 2018  
         7  Grand Slam Abu Dhabi 2018  
         8  Grand Slam Abu Dhabi 2018  
         9  Grand Slam Abu Dhabi 2018  
\end{Verbatim}
            
    \begin{Verbatim}[commandchars=\\\{\}]
{\color{incolor}In [{\color{incolor}11}]:} \PY{n+nb}{print}\PY{p}{(}\PY{n}{f}\PY{l+s+s1}{\PYZsq{}}\PY{l+s+s1}{Es decir, un total de }\PY{l+s+si}{\PYZob{}df.shape[0]\PYZcb{}}\PY{l+s+s1}{ acciones en }\PY{l+s+s1}{\PYZob{}}\PY{l+s+s1}{df[[}\PY{l+s+s1}{\PYZdq{}}\PY{l+s+s1}{opponent}\PY{l+s+s1}{\PYZdq{}}\PY{l+s+s1}{, }\PY{l+s+s1}{\PYZdq{}}\PY{l+s+s1}{event}\PY{l+s+s1}{\PYZdq{}}\PY{l+s+s1}{]].drop\PYZus{}duplicates().shape[0]\PYZcb{} batallas}\PY{l+s+s1}{\PYZsq{}}\PY{p}{)}
\end{Verbatim}


    \begin{Verbatim}[commandchars=\\\{\}]
Es decir, un total de 99 acciones en 29 batallas

    \end{Verbatim}

    Agrupamos por sujeto y acción

    \begin{Verbatim}[commandchars=\\\{\}]
{\color{incolor}In [{\color{incolor}12}]:} \PY{n}{grouped\PYZus{}actions} \PY{o}{=} \PY{n}{df}\PY{o}{.}\PY{n}{groupby}\PY{p}{(}\PY{p}{[}\PY{l+s+s1}{\PYZsq{}}\PY{l+s+s1}{you}\PY{l+s+s1}{\PYZsq{}}\PY{p}{,} \PY{l+s+s1}{\PYZsq{}}\PY{l+s+s1}{action}\PY{l+s+s1}{\PYZsq{}}\PY{p}{]}\PY{p}{)}\PY{o}{.}\PY{n}{size}\PY{p}{(}\PY{p}{)}
         \PY{n}{grouped\PYZus{}actions}
\end{Verbatim}


\begin{Verbatim}[commandchars=\\\{\}]
{\color{outcolor}Out[{\color{outcolor}12}]:} you  action         
         0    Cancel Shido        1
              HSK (3rd shido)     5
              Ippon               6
              Shido              24
              Waza-ari            6
         1    Cancel Ippon        2
              Cancel Waza-ari     1
              HSK (3rd shido)     4
              Ippon               1
              Shido              38
              Waza-ari           11
         dtype: int64
\end{Verbatim}
            
    \begin{Verbatim}[commandchars=\\\{\}]
{\color{incolor}In [{\color{incolor}13}]:} \PY{n+nb}{print}\PY{p}{(}\PY{n}{f}\PY{l+s+s1}{\PYZsq{}}\PY{l+s+s1}{Por ejemplo, hiciste }\PY{l+s+si}{\PYZob{}grouped\PYZus{}actions[1, \PYZdq{}Ippon\PYZdq{}]\PYZcb{}}\PY{l+s+s1}{ Ippones y recibiste }\PY{l+s+si}{\PYZob{}grouped\PYZus{}actions[1, \PYZdq{}Shido\PYZdq{}]\PYZcb{}}\PY{l+s+s1}{ Shidos}\PY{l+s+s1}{\PYZsq{}}\PY{p}{)}
\end{Verbatim}


    \begin{Verbatim}[commandchars=\\\{\}]
Por ejemplo, hiciste 1 Ippones y recibiste 38 Shidos

    \end{Verbatim}

    \subsubsection{Detail Gaitero}\label{detail-gaitero}

    \begin{Verbatim}[commandchars=\\\{\}]
{\color{incolor}In [{\color{incolor}14}]:} \PY{n}{df}\PY{o}{.}\PY{n}{groupby}\PY{p}{(}\PY{p}{[}\PY{l+s+s1}{\PYZsq{}}\PY{l+s+s1}{you}\PY{l+s+s1}{\PYZsq{}}\PY{p}{,} \PY{l+s+s1}{\PYZsq{}}\PY{l+s+s1}{action\PYZus{}detail}\PY{l+s+s1}{\PYZsq{}}\PY{p}{]}\PY{p}{)}\PY{o}{.}\PY{n}{size}\PY{p}{(}\PY{p}{)}
\end{Verbatim}


\begin{Verbatim}[commandchars=\\\{\}]
{\color{outcolor}Out[{\color{outcolor}14}]:} you  action\_detail                     
         0    Ashi-waza / Ko-uchi-gari               1
              Ashi-waza / O-uchi-gari                2
              Ashi-waza / Uchi-mata                  1
              Koshi-waza / Sode-tsurikomi-goshi      1
              Koshi-waza / Tsuri-goshi               1
              Osaekomi-waza / Kesa-gatame            1
              Osaekomi-waza / Ura-gatame             1
              Osaekomi-waza / Yoko-shiho-gatame      1
              Shido / Avoid-Grip                     1
              Shido / Defensive-Posture              2
              Shido / False-Attack                  11
              Shido / Hold-Same-Side                 1
              Shido / Non-Combativity               11
              Shido / Outside-Contest-Area           4
              Te-waza / Seoi-nage                    1
              Te-waza / Tai-otoshi                   1
              Yoko-sutemi-waza / O-soto-makikomi     1
         1                                           4
              Ashi-waza / O-uchi-gaeshi              1
              Ashi-waza / Uchi-mata-gaeshi           1
              Koshi-waza / Sode-tsurikomi-goshi      1
              Ma-sutemi-waza / Sumi-gaeshi           3
              Osaekomi-waza / Tate-shiho-gatame      1
              Shido / Avoid-Grip                     6
              Shido / Defensive-Posture              3
              Shido / Escape-With-Head               1
              Shido / False-Attack                   6
              Shido / Fingers-In-Sleeve              1
              Shido / Fingers-interlocked            1
              Shido / Hold-Sleeve-Ends               1
              Shido / Hold-Trouser-Leg               1
              Shido / Non-Combativity               17
              Shido / Outside-Contest-Area           2
              Shido / Pistol-Grip                    1
              Shido / Undetermined                   1
              Te-waza / Ippon-seoi-nage              1
              Te-waza / Uchi-mata-sukashi            1
              Yoko-sutemi-waza / Tani-otoshi         3
         dtype: int64
\end{Verbatim}
            

    % Add a bibliography block to the postdoc
    
    
    
    \end{document}
